\documentclass[12pt,a4paper]{amsart}
% ukazi za delo s slovenscino -- izberi kodiranje, ki ti ustreza
\usepackage[slovene]{babel}
\usepackage[utf8]{inputenc}
%\usepackage[T1]{fontenc}
%\usepackage[utf8]{inputenc}
\usepackage{amsmath,amssymb,amsfonts}
\usepackage{url}
%\usepackage[normalem]{ulem}
\usepackage[dvipsnames,usenames]{color}
\usepackage{algpseudocode}
\usepackage{graphicx}
\usepackage[a-1b]{pdfx}


% ne spreminjaj podatkov, ki vplivajo na obliko strani
\textwidth 15cm
\textheight 24cm
\oddsidemargin.5cm
\evensidemargin.5cm
\topmargin-5mm
\addtolength{\footskip}{10pt}
\pagestyle{plain}
\overfullrule=15pt % oznaci predlogo vrstico


% ukazi za matematicna okolja
\theoremstyle{definition} % tekst napisan pokoncno
\newtheorem{definicija}{Definicija}[section]
\newtheorem{opomba}[definicija]{Opomba}

\theoremstyle{definition}
\newtheorem{example}{Zgled}

\theoremstyle{definition}
\newtheorem{dokaz}{Dokaz}

\theoremstyle{definition}
\newtheorem{algoritem}{Algoritem}

%\renewcommand\endprimer{\hfill$\diamondsuit$}


\theoremstyle{plain} % tekst napisan posevno
\newtheorem{lema}[definicija]{Lema}
\newtheorem{izrek}[definicija]{Izrek}
\newtheorem{trditev}[definicija]{Trditev}
\newtheorem{posledica}[definicija]{Posledica}


% za stevilske mnozice uporabi naslednje simbole
\newcommand{\R}{\mathbb R}
\newcommand{\N}{\mathbb N}
\newcommand{\Z}{\mathbb Z}
\newcommand{\C}{\mathbb C}
\newcommand{\Q}{\mathbb Q}

\newcommand{\al}{\alpha}
\newcommand{\g}{\triangledown}
\newcommand{\abs}{\Vert}


% ukaz za slovarsko geslo
\newlength{\odstavek}
\setlength{\odstavek}{\parindent}
\newcommand{\geslo}[2]{\noindent\textbf{#1}\hspace*{3mm}\hangindent=\parindent\hangafter=1 #2}


% naslednje ukaze ustrezno popravi
\newcommand{\program}{Finančna matematika} % ime studijskega programa: Matematika/Finan'cna matematika
\newcommand{\imeavtorja}{Tia Krofel, Brina Ribič, Matej Rojec} % ime avtorja
\newcommand{\imementorja}{dr. Aleš Ahčan} % akademski naziv in ime mentorja
\newcommand{\naslovdela}{VaR on option portfolio}
\newcommand{\letnica}{2022}


% vstavi svoje definicije ...




\begin{document}

% od tod do povzetka ne spreminjaj nicesar
\thispagestyle{empty}
\noindent{\large
UNIVERZA V LJUBLJANI\\[1mm]
FAKULTETA ZA MATEMATIKO IN FIZIKO\\[5mm]
\program\ -- 2.~stopnja}
\vfill

\begin{center}{\large
\imeavtorja\\[2mm]
{\bf \naslovdela}\\[10mm]
Seminarska naloga pri predmetu Upravljanje tveganj\\[1cm]
Mentor: \imementorja}
\end{center}
\vfill

\noindent{\large
Ljubljana, \letnica}
\pagebreak

\thispagestyle{empty}
\tableofcontents
\pagebreak

\section{Osnovno o opcijah}
% B-S model
% grki
% zaščitni portfelj

\section{Osnovno o VaR}

"Value at Risk" oziroma VaR je mera, ki je opredeljena kot največja potencialna sprememba v vrednosti portfelja pri določeni,
dovolj visoki stopnji zaupanja za vnaprej določeno časovno obdobje. Ponavadi je stopnja zaupanja $95\%$ ali $99\%$.
VaR nam pove, koliko lahko izgubim z $x\%$ verjetnostjo v nekem časovnem obdobju. Ponavadi se uporablja
krajše časovno obdobje, recimo dan, teden ali nekaj tednov.
To pomeni, če je VaR za neko sredstvo $100$ milijonov evrov v obdobju enega tedna s stopnjo zaupanja $95\%$,
potem je samo $5\%$ verjetnost, da bo vrednost sredstva padla za več kot $100$ milijonov evrov v katerem koli tednu.

Pri opcijah pa ni tako preprosto, saj mora ocena tveganja  upoštevati nelinarno gibanje cen (gamma učinek)
in posredna volatilnost (vega učinek). % implied volatilities
Za opcije bomo nelinearno gibanje cen ocenili analitično (delta-gamma) ali s simulacijo. 

\subsection{Kako izračunamo VaR?}

Obstajajo trije osnovni pristopi, kako izračunati VaR. Lahko jo izračunamo analitično
s predpostavkami o porazdelitvah donosov za tržna tveganja, zraven pa moramo upoštevati variance
in kovariance med temi tveganji. VaR lahko ocenimo tudi s hipotetičnim portfeljem preko historičnih 
podatkov ali z Monte Carlo simulacijo. 

\subsubsection{Mean variance framework}

\subsubsection{Historical simulation}
% tega načina za opcije ne moremo uporabit, ker je težko zbrati historične podatke za opcije
% izpostavljenost bi morala biti pogojevana na veliko
% karakteristik, da bi lahko izračunali standardni var
% te so: volatilnost, čas dospelosti, S_t, strike price
% če to vse vstavimo v B-S model postane model nelinear zaradi narave derivativov


\subsubsection{Monte Carlo simulacija}
% uporabno, ker lahko upoštevamo nelinarnost

\subsection{Prednosti in slabosti}
% tu je mišljeno tega navadnega -> zato vpeljemo non-linear
% pri derivativih imamo nelinearno izpostavljenost k tveganju -> Kaj to pomeni?

\section{Nonlinear VaR}
% uporablja se za derivative
% tiste FI, ki imajo nelinearno izpostavljenost k tveganju
% mapping!
% B-S formula, od tod potem sledi da je izpostavljenost nelinearna
% analytical approximation and structured monte carlo simulation
VaR meri tržno tveganje portfelja. Tu si portfelj predstavljamo kot množico 
pozicij, od katerih je vsaka sestavljena iz osnovnih vrednostnih papirjev. Zato razvrstimo pozicije
v preprostejše, ki so linearne, in v pozicije izvedenih finančnih instrumentov, ki jih lahko še naprej
razdelimo v linearne ali nelinearne pozicije izvedenih finančnih instrumentov.

% primer preproste pozicije

Osnovna različica VaR predpostavlja linearno povezavo med donosi in spremembo vrednosti pozicije.
Tu predpostavljamo, da imajo donosi vrednostnega papirja večrazsežno pogojno normalno porazdelitev.  

Pri opcijskih pozicijah je nelinearna povezava med spremembo vrednosti pozicije in donosom. 
To lahko pojasnimo s preprosto opcijo na delnico. Cena opcije je $V(P_t,K,\tau,\rho,\sigma)$ v 
odvisnosti od cene delnice $P_t$ ob času $t$, izvršilne cene $K$, časa dospelosti $\tau$. 
Cena opcije je odvisna tudi od netvegane obrestne mere $\rho$ nekega vrednostnega papirja, ki ima enak
čas dospelosti kot opcija ter od standardnega odklona $\sigma$ logaritma cene delnice v časovnem obdobju opcije.

Pri delta-gama pristopu še vedno predpostavljamo, da so donosi vrednostnih papirjev normalno porazdeljeni. 
Dodatno dopuščamo nelinearno zvezo med vrednostjo pozicije in donosi temelja. Natančneje, dovoljujemo gama učinek,
torej da relativna sprememba portfelja iz derivativov (v našem primeru opcij) ni več normalno porazdelja. 
Zaradi tega ne moremo več VaR definirati kot 1.65 krat standardni odklon portfelja. Namesto tega VaR 
izračunamo v dveh glavnih korakih. Najprej izračunamo prve štiri momente porazdelitve donosa portfelja, tj.,
povprečje, standardni odklon, \textbf{skewness}, \textbf{kurtosis}. Potem poiščemo 
porazdelitev, ki ima enake prve štiri momente kot porazdelitev donosa portfelja in izračunamo peti 
percentil (ali prvi, odvidno od problema). Od tod potem dobimo VaR.


\end{document}